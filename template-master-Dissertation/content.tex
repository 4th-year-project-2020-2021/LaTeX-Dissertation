%!TEX root = project.tex

\chapter*{About this project}
\paragraph{Abstract}
Data visualisation has become popular, in part due to everyone's experience with the global pandemic Covid-19. Information is delivered in easy to understand visual representations. As this is an emerging area, the team decided to investigate different methods of data visualisation, see which performed the best for different scenarios and to develop a data visualisation application. The main focus of the application is on Covid-19, data is being generated at least daily worldwide for Covid-19. The application includes examples of other pandemics such as SARs, MERs and historic epidemics of smallpox and polio.

\paragraph{Authors}
This project was developed as part of a fifteen credit module by Grace Keane, Jina Kim and Shirin Nagle, fourth year students of Galway Mayo Institute of Technology.

\paragraph{Acknowledgements} The authors would like to acknowledge the time and advice given by the project supervisor Dr. John French.

\chapter{Introduction}
This chapter will serve as a prelude to the project. The context, objectives and metrics for success and failure will be defined, followed by a thorough summary of the project domain and an introduction into each chapter of the dissertation and its relevance.

\section{Context}
During the decision making process the team decided that the project must be relevant as well as interesting. The project must encourage the use of existing skills as well as allow for the natural development of new and relevant techniques and processes while also being worthy of scope.

\vspace{5mm} %5mm vertical space

Data visualisation and analysis techniques have been the front and center in the efforts to communicate the statistics as well as the science around the COVID-19 virus. Interactive dashboards with several charts and graphs surfaced in different formats to offer concise ways to make sense of complex and overwhelming pandemic data sets. These techniques have become essential in informing the general public as well as healthcare providers, scientists and governments of the overall COVID-19 growth. 

\vspace{5mm} %5mm vertical space

As a result, the team felt it would be beneficial as well as informative to create a web application that would take large COVID-19 data sets as well as other pandemic data and create a series of data visualisations from it. As well as compare past and present viruses to distinguish similarities along with differences. This would provide a clear view of COVID-19 and how similar viruses have spread.

\vspace{5mm} %5mm vertical space

Once the project area had been decided on, each team member focused on a pandemic or epidemic, investigated where to source relevant information for the pandemic or epidemic, and investigated different technologies for displaying data in a visual manner.

\section{Project Objectives}
As previously mentioned, the end goal of the project is to create a web application that would process different virus data and create various data visualisations to ensure a simple way for users to view large complex pandemic data. To ensure the end goal was reached, the team outlined objectives to be followed post development.

\begin{itemize}
  \item Investigate the field of interest and scope for the project.
  \item Evaluate and research the appropriate frameworks and tools available for development.
  \item Incorporate state of the art technologies, frameworks, databases and tools that will allow users to view, add, update, delete, post and retrieve data.
  \item The web application will, at a minimum allow users to sign-up, login, view data visualizations as well as incorporate user interaction features, create unique user visuals and post user data to databases. 
\end{itemize}

\subsection{Metrics for Success and Failure}
The outlining criteria for success and failure were important in ensuring the project remained on track throughout development and therefore allow the project to achieve the outlined goals in a timely and standard-driven manner. The defined metrics are of similar nature to that of the underlying objectives defined in \emph{Section .....} but explained at a higher level.

\vspace{5mm} %5mm vertical space

\subsection{Applied Project Dissertation}
A selection of metrics were outlined specifically for the dissertation, they are as follows.

\begin{itemize}

    \item \emph{The final dissertation should be easily understood, allowing readers without knowledge or familiarity of the subject area to form a suitable understanding of the concept and areas discussed.} In order to accurately measure this metric, input was received from family and fellow students, who would kindly read newly integrated sections and provide useful feedback.

    \item \emph{The  dissertation should be in order of 30,000 - 45,000 words, excluding appendices as well as approximately 105 pages (excluding diagrams, abstract, TOCs, references, appendices etc.)}
    
\end{itemize}

\section{Applied Project}
A selection of metrics were outlined specifically for the development of the applied project, they were as follows.

\begin{itemize}

    \item \emph{The applied project must be easy to use, navigate and attempt to deal with a task of problem deemed to be of sufficient technical challenge and depth.} To ensure this was adhered to, at frequent stages of development feedback from fellow students and Dr. John French was considered and documented relating to the application.

    \item \emph{Collaboration between team members and supervisors should be consistent to ensure a smooth work flow throughout the project planning and development.)} In order for this metric to be adhered to, the team maintained communication with themselves and Dr John French throughout the development and planning of the applied project.
    
\end{itemize}

\section{Dissertation Summary}
This section will contain a brief overview of the dissertation structure and will provide a short description of each chapters objectives.

\section{Methodology}
In this chapter, the processes undertaken during the life cycle of the project regarding planning, testing and development will be outlined. Investigating development methods and methods employed during the research phase, their results and the effect they had on the direction of the project will be brought to the attention of the reader. Additional, the decisions, thought process and influential factors leading up to those processes and design implementations will also be described.

\section{Technology Review}
A technological review will encapsulate the technical aspect of the project. This includes the different technologies incorporated, their implementation, their roll, and why they were chosen. The benefits of the chosen technologies will be critically analysed and compared with similar alternatives. 

\section{System Design}
A detailed explanation of the overall system architecture will be provided. Diagrams will be included to help illustrate and explain the inner workings of the application at a high level.

\section{System Evaluation}
An evaluation of the final applied project against the initial project objectives. The final result of the project will be critically analysed including an analysis of areas of software quality, improvements or changes to the overall project.

\section{Conclusion}
To conclude, a brief summary of the context and objectives to remind the reader about the overall rationale and goals of the project. Key insights will be identified and reflected on. A final analysis will describe the teams overall experience and what the team learned while working on a project similar to one encountered in the software industry. 


\chapter{Methodology}
The first project meeting took place in the final week of September 2020. This meeting consisted of analysing the applied project requirements that had been defined and outlined. The team agreed to choose a topic that would be informative yet beneficial to the end user. It was concluded that the final idea and architecture of the solution should be finalised as soon as possible to allow for necessary research and planned development.

\vspace{5mm} %5mm vertical space

This chapter will explore the pre-development process, the approach the team took with development and testing, how the team dealt with various problems faced, and the influence of regular supervisor meetings along with a conclusion based what impact pre-development research had on the overall project direction. 

\section{Technological Brainstorming and Initial Supervisor Meeting}
Before development began and after the initial project topic was chosen each team member researched various technologies and concepts that could be potentially incorporated into the project. Soon after a brainstorming meeting was conducted to critically analyse each technology researched and the team decided whether each would be of good fit.

\vspace{5mm} %5mm vertical space

The team met prior to the initial supervisor meeting to produce effective ideas and questions to be asked relating to the overall architecture of the solution and the ultimate goals and objectives, these included:

\begin{itemize}

    \item \textbf{What research areas should be prioritised before development is conducted?}

    \item \textbf{What technologies would be of best fit for the project?}
    
    \item \textbf{What benefits would certain methodologies posses compared to others?}
    
\end{itemize}

During the initial supervisor meeting the project idea, potential research, development technologies, development methodology and collaboration approaches were discussed. The team were advised to spend the next few weeks to consider and finalise the overall nature of how the applied project will be designed and implemented.

\vspace{5mm} %5mm vertical space

After the supervisor meeting the team decided to take this advise on board. Each member focused on researching the What? Why? How? of the project technology, idea and the overall end user experience and relevance of the project. Once research was carried out the team had a team meeting to discuss and critically analyse the findings. Some questions asked were as follows:

\begin{itemize}

    \item \textbf{What development methodology and collaboration technique would be best to adapt?}
    
    \item \textbf{Would the chosen project idea be of relevance to the end user?}
    
     \item \textbf{Would the chosen technologies be of a professional standard?}
    
    \item \textbf{How would the chosen technologies fit together as a whole?}
    
    \item \textbf{Is the scope an appropriate size for a three person project?}
    
\end{itemize}

\section{Methodology Consideration}
There were numerous possible methodologies to consider. After discussions with team members, the team agreed that using \textbf{Waterfall} and \textbf{Agile} together would be of most benefit to the applied project progression. 


\vspace{70mm} %5mm vertical space


\section{Determining Development Methodology}
Following the decision to use both Agile and Waterfall as an approach to software development, the team began to research the advantages and disadvantages of both, along with deciding which methodology would be best to take when considering the scope, project goals and time frame. A critical analysis and overview of both methodologies were made and a conclusion based on the practical analysis of each in relation to the project was undertaken.

\section{Waterfall}
The Waterfall model is composed of a series of steps, illustrated in \emph{Figure....}. These steps are used to break down a project into linear sequential phases to ease the development process.  \cite{petersen2009waterfall}. Waterfall allows progress to be easily measured, the complete scope of the project is known in advance, which can be preferred based on the type of project and software team. The Waterfall model tends to be among the less flexible approaches, the output of each stage completed is the input for the next and the overall progress follows a fixed downwards path like a waterfall.

\begin{center}    
      \includegraphics[scale=0.46]{img/Waterfall.png}
\end{center}
\label{fig:x cubed graph}

The waterfall model is still a widely used method for working on software projects today. It is arguably the most well known development model, which is likely attributed to how long the model has been around, not to mention the overall simplicity of the model. Each phase is of a waterfall project is completed sequentially and testing is done after tasks are completed. \cite{balaji2012waterfall}. 


\section{Agile}
Agile methodologies are a group of software development methods that are based on interactive and incremental development \cite{kumar2012impact}. The term agile stands for 'moving quickly'. It is characterised by an approach that allows rapid and continuous delivery of small and useful software. It takes the view  that production teams should start with simple and predictable approximations to the final requirement and then continue to increment the detail of these requirements throughout the life of the project. Agile methodology has an adaptive team which is able to respond to changing requirements as well as the ability to solve problems even late in the development cycle. \cite{balaji2012waterfall}.
\begin{center}    
      \includegraphics[scale=0.8]{img/Agile.png}
\end{center}
The popularity of Agile development methodologies amongst the software industry has increased drastically, with almost 85.4\% of international surveyed software developers using Agile methodologies in their work. \cite{agileStats} Research has also shown that Agile projects are 28\% more successful than traditional projects. The most important principle of the Agile methodology is customer satisfaction by giving rapid and continuous delivery of small and useful software.

\section{Comparison between Agile and Waterfall}
Following team meetings, research and analysis, both Agile and Waterfall were critically analysed under the following headings:

\begin{itemize}

    \item \textbf{Requirement delivery}
    
    \item \textbf{Flexibility}
    
    \item \textbf{Compatibility}
    
     \item \textbf{Methodology past experience}
    
\end{itemize}

By taking the above headings into an account comparisons were drawn of both methodologies.



***Need to create a table here****

waterfall


\begin{itemize}

    \item \textbf{Requirements are clear before development starts.}
    
    \item \textbf{Each development phase is completed in a set time frame. Then it moves to another phase. Which leads to easy}
    
    \item \textbf{It is a linear model so therefore easy to implement.}
    
    \item \textbf{Requirements, Specification and Design can take up a lot of time.}
     
    \item \textbf{Low flexibility meaning it may be difficult of even impossible to make major changes while in the implementation phase.}
    
\end{itemize}



agile


\begin{itemize}

    \item \textbf{Ability to respond to the changing requirements of the project.}
    
    \item \textbf{Errors can be detected and solved quickly.}
    
    \item \textbf{There is constant communication and inputs from the team and supervisor which aids the development process.}
     
    \item \textbf{Project can easily fall off track.}
     
    \item \textbf{End point of the project if not clearly defined.}
     
    \vspace{70mm} %5mm vertical space
 
    
\end{itemize}

Following various comparisons being drawn between the Agile and Waterfall, it became clear to the team that the Agile approach would best suit the development of the outlined applied project. This approach would aid the development process in numerous ways, these include:

\vspace{70mm} %5mm vertical space

1. \textbf{Errors can be detected and solved quickly} - The team thought it would be beneficial to be able to detect and solve problems quickly, especially when the team did not have much experience with using GitHub collaboratively together as a team.

2. \textbf{The ability to develop small but functional software via releases} - Based on feedback from our supervisor the priorities of the project would be subject to change. Small incremental changes means there would be flexibility and versatility throughout the development stage.

3. \textbf{GitHub project boards and Scrum Methodologies} - The idea of incremental releases in parallel with scrum sprints based on workflow defined in GitHub project boards was something the team thought would be of major benefit to the overall workflow.





\vspace{70mm} %5mm vertical space

About one to two pages.
Describe the way you went about your project:
\begin{itemize}
\item Agile / incremental and iterative approach to development. Planning, meetings.
\item What about validation and testing? Junit or some other framework.
\item If team based, did you use GitHub during the development process.
\item Selection criteria for algorithms, languages, platforms and technologies.
\end{itemize}
Check out the nice graphs in Figure \ref{tikz:graphs}, and the nice diagram in Figure \ref{tikz:mydiagram}.



\chapter{Technology Review}
About seven to ten pages.
\begin{itemize}
\item Describe each of the technologies you used at a conceptual level. Standards, Database Model (e.g. MongoDB, CouchDB), XMl, WSDL, JSON, JAXP.
\item Use references (IEEE format, e.g. [1]), Books, Papers, URLs (timestamp) – sources should be authoritative. 
\end{itemize}

\section{XML}
Here's some nicely formatted XML:
\begin{minted}{xml}
<this>
  <looks lookswhat="good">
    Good
  </looks>
</this>
\end{minted}

\chapter{System Design}
FReMP stack is a very powerful and highly scalable stack which reduces complexity of back-end code, deals with the database requests very quickly and brings a user friendly interface for front-end users.

\begin{center}    
      \includegraphics{img/fremp.PNG}
\end{center}

\begin{itemize}
\item Flask
\item React.js
\item MongoDB Atlas
\item Python

These technologies were chosen as they give developers the ability to handle huge volume of data and, at the same time, render a nice modern interface on the client side. 
\item Architecture, UML etc. An overview of the different components of the system. Diagrams etc… Screen shots etc.
\end{itemize}

\begin{table}[h]
  \centering
  \begin{tabular}{x{2cm}p{3cm}}
    \toprule \\
    Column 1 & Column 2 \\
    \midrule \\
    Rows 2.1 & Row 2.2 \\
    \bottomrule
  \end{tabular}
  \caption{A table.}
  \label{table:mytable}
\end{table}

\chapter{System Evaluation}
As many pages as needed.
\begin{itemize}
\item Prove that your software is robust. How? Testing etc. 
\item Use performance benchmarks (space and time) if algorithmic.
\item Measure the outcomes / outputs of your system / software against the objectives from the Introduction.
\item Highlight any limitations or opportuni-ties in your approach or technologies used.
\end{itemize}

\chapter{Conclusion}
About three pages.

\begin{itemize}
\item Briefly summarise your context and ob-jectives (a few lines).
\item Highlight your findings from the evalua-tion section / chapter and any opportuni-ties identified.
\end{itemize}



\heading
